\documentclass{standalone}
% preamble: usepackage, etc.
\begin{document}
\chapter{Conclusion}
\label{Chapter5}
Due to the obvious increased output of large amounts of information in recent times, particularly images, it is becoming necessary to build techniques and software for the automatic collection, processing, and analysis of such information in order to extract trends and patterns. This guarantees that images and its characteristics are converted by algorithm into representations that reflect their context and interpretation. Currently, innovations in artificial intelligence are being applied to system algorithms for image analysis and complicated analysis in the field of machine vision. Image classification and recognition are essential components of machine vision. 


\section{Research Findings}
In this dissertation, we examined the application of wavelet multi-resolution analysis and deep learning models as a single framework for the task of machine fault diagnosis, biometric identification and disease classification. 

\section{Contributions}
In this dissertation, we presented wavelet multi-resolution analysis algorithm and deep learning model as a single framework capable of attaining and advancing the horizon of research in medical imaging, machine fault diagnosis and biometric identification. 

In Chapter 2, We analyzed and presented the impact of existing approaches using both classic and cutting-edge techniques.

In Chapter 3, we developed an  image classification and identification for the feature extraction and classification of images.

In Chapter 4, we presented an improved capsule network for accurately classifying images. The images becomes the input data to the wavelet capsule network for feature extraction and classification.


\section{Future Works}
Given the vast amount of research in vision recognition and image classification, there are still a number of obstacles to overcome before reaching same accuracy of that of human. To begin, our models are supervised framework, which necessitates a large quantity of data to learn the mapping from an input $X$ to a label $Y$. 

As a result, achieving unsupervised artificial general intelligence appears to be a promising study area and natural next step in the advancement of machine intelligence in all areas.

\end{document}